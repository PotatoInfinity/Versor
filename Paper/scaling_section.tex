\subsection{Manifold Capacity and Multi-Channel Scaling}
A central bottleneck in geometric sequence modeling is \textit{manifold crowding}. In single-channel architectures (such as the standard Versor RNN), all interaction logic for $N$ particles must be compressed into a single $\Spin$ rotor. As $N$ increases, the probability of "geometric interference" grows, limiting the model's capacity to resolve independent force vectors.

To demonstrate this, we stress-test the architecture on a 15-body gravitational system---a 3$\times$ increase in density over the original task. We compare the Single-Channel RNN baseline (109k parameters) against a significantly disadvantaged Multi-Channel Versor model (48k parameters).

\begin{table}[H]
\centering
\caption{Density Stress Test ($N=15$ bodies). Despite having 2.3$\times$ fewer parameters, the Multi-Channel architecture outperforms the Single-Channel RNN, proving that parallel Clifford channels are more efficient at resolving dense interaction manifolds than increasing the size of a single temporal state.}
\begin{tabular}{lccl}
\toprule
Model Architecture & Params & MSE ($\downarrow$) & Result \\
\midrule
Single-Channel RNN (Baseline) & 109k & 12.81 & Baseline \\
Multi-Channel (Versor)        & \textbf{48k}  & \textbf{12.68} & \textbf{+1.0\% Improvement} \\
\bottomrule
\end{tabular}
\end{table}

The Multi-Channel architecture mitigates crowding by assigning internal features to orthogonal Clifford subspaces. This parallel representation allows the model to track independent sub-clusters of the planetary system simultaneously. The fact that the 48k model outperforms the 109k baseline confirms that \textit{Width} (Channels) is more critical than \textit{Depth} for high-density physical sequence modeling.
